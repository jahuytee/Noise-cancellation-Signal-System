% Audio DSP Project Lab Report
% Noise Reduction and Active Noise Cancellation System

\documentclass[12pt,letterpaper]{article}

% Packages
\usepackage[margin=1in]{geometry}
\usepackage{graphicx}
\usepackage{amsmath}
\usepackage{amssymb}
\usepackage{hyperref}
\usepackage{listings}
\usepackage{xcolor}
\usepackage{float}
\usepackage{subcaption}
\usepackage{booktabs}
\usepackage{siunitx}

% Code listing style
\lstset{
    language=Python,
    basicstyle=\ttfamily\small,
    keywordstyle=\color{blue},
    commentstyle=\color{gray},
    stringstyle=\color{red},
    numbers=left,
    numberstyle=\tiny\color{gray},
    breaklines=true,
    frame=single,
    backgroundcolor=\color{gray!10}
}

% Title information
\title{Real-Time Noise Reduction and Simulated Active Noise Cancellation System}
\author{Jason Huynh}
\date{\today}

\begin{document}

\maketitle

\begin{abstract}
This report documents the design, implementation, and testing of a comprehensive audio digital signal processing (DSP) system. The project consists of two main components: (1) a real-time noise reduction system using software-based filtering techniques including band-pass filtering, notch filtering, spectral subtraction, and Wiener filtering, and (2) a simulated Active Noise Cancellation (ANC) system implementing the Filtered-x Least Mean Squares (FxLMS) adaptive algorithm. The system processes audio from a laptop microphone at 44.1 kHz sampling rate, demonstrating fundamental DSP concepts and practical noise reduction techniques. Results show significant SNR improvements across multiple filtering methods, with the combined approach achieving [X] dB improvement in typical noisy environments.
\end{abstract}

\tableofcontents
\newpage

% ============================================================================
\section{Introduction}

\subsection{Project Overview}
This project implements a complete audio noise reduction pipeline, demonstrating both classical filtering techniques and modern adaptive algorithms. The system is designed to:

\begin{itemize}
    \item Capture and process real-time audio from a microphone
    \item Visualize audio signals in both time and frequency domains
    \item Apply various noise reduction techniques
    \item Quantitatively measure improvements using SNR metrics
    \item Simulate active noise cancellation without hardware requirements
\end{itemize}

\subsection{Motivation}
Noise reduction is a fundamental challenge in audio signal processing, with applications in telecommunications, hearing aids, audio recording, and consumer electronics. This project provides hands-on experience with:

\begin{itemize}
    \item Fourier analysis and frequency-domain processing
    \item Digital filter design (IIR and FIR)
    \item Adaptive filtering algorithms
    \item Real-time signal processing constraints
    \item Audio quality metrics
\end{itemize}

\subsection{Project Structure}
The implementation is divided into two phases:

\textbf{Phase A: Real-Time Noise Reduction}
\begin{itemize}
    \item A0: Minimal pipeline (audio I/O, FFT, visualization)
    \item A1: Baseline filters (band-pass, notch)
    \item A2: Advanced noise reduction (spectral subtraction, Wiener)
    \item A3: Command-line interface integration
\end{itemize}

\textbf{Phase B: Simulated Active Noise Cancellation}
\begin{itemize}
    \item B0: ANC simulation environment
    \item B1: FxLMS algorithm implementation
    \item B2: Parameter optimization experiments
\end{itemize}

% ============================================================================
\section{Theoretical Background}

\subsection{Digital Signal Processing Fundamentals}

\subsubsection{Sampling Theory}
Audio signals are sampled at $f_s = 44100$ Hz, satisfying the Nyquist criterion:
\begin{equation}
    f_s \geq 2 f_{max}
\end{equation}
where $f_{max} \approx 20$ kHz is the upper limit of human hearing.

\subsubsection{Fast Fourier Transform}
The Discrete Fourier Transform (DFT) converts time-domain signals to frequency domain:
\begin{equation}
    X[k] = \sum_{n=0}^{N-1} x[n] e^{-j2\pi kn/N}
\end{equation}
Implemented efficiently using the FFT algorithm with $O(N \log N)$ complexity.

\subsection{Filtering Techniques}

\subsubsection{Band-Pass Filter}
A 5th-order Butterworth band-pass filter isolates speech frequencies (80--8000 Hz):
\begin{itemize}
    \item Passband: 80--8000 Hz
    \item Stopband attenuation: $>$ 40 dB
    \item Zero-phase implementation using \texttt{filtfilt()}
\end{itemize}

The transfer function for a Butterworth filter:
\begin{equation}
    |H(j\omega)|^2 = \frac{1}{1 + \left(\frac{\omega}{\omega_c}\right)^{2n}}
\end{equation}

\subsubsection{Notch Filter}
IIR notch filters remove narrow-band interference at 60 Hz and harmonics:
\begin{equation}
    H(z) = \frac{1 - 2\cos(\omega_0)z^{-1} + z^{-2}}{1 - 2r\cos(\omega_0)z^{-1} + r^2z^{-2}}
\end{equation}
where $\omega_0$ is the notch frequency and $r$ determines the bandwidth (Q-factor).

\subsubsection{Spectral Subtraction}
Noise spectrum is estimated during silent periods and subtracted:
\begin{equation}
    |\hat{X}[k]| = \max\left(|Y[k]| - \alpha|\hat{N}[k]|, \beta|Y[k]|\right)
\end{equation}
where $\alpha$ is the over-subtraction factor and $\beta$ is the spectral floor.

\subsubsection{Wiener Filter}
Optimal filter in minimum mean-square error sense:
\begin{equation}
    H[k] = \frac{S_{xx}[k]}{S_{xx}[k] + S_{nn}[k]}
\end{equation}
where $S_{xx}$ is signal PSD and $S_{nn}$ is noise PSD.

\subsection{Active Noise Cancellation}

The FxLMS algorithm adapts filter weights to minimize error:
\begin{align}
    e[n] &= d[n] + y[n] * s[n] \\
    w[n+1] &= w[n] + \mu \cdot e[n] \cdot x_f[n]
\end{align}
where $x_f[n] = x[n] * \hat{s}[n]$ is the filtered reference signal.

% ============================================================================
\section{Implementation}

\subsection{System Architecture}

The system is implemented in Python with the following components:

\begin{itemize}
    \item \texttt{stream.py}: Real-time audio I/O and visualization
    \item \texttt{utils.py}: Signal analysis utilities (SNR, plotting)
    \item \texttt{filters.py}: Filter implementations
    \item \texttt{spectral\_subtraction.py}: Spectral subtraction algorithm
    \item \texttt{wiener.py}: Wiener filter implementation
    \item \texttt{fxlms.py}: FxLMS adaptive algorithm
\end{itemize}

\subsection{Software Stack}
\begin{itemize}
    \item \textbf{Python 3.10}: Primary language
    \item \textbf{NumPy 1.26.4}: Array operations, FFT
    \item \textbf{SciPy 1.15.3}: Filter design, signal processing
    \item \textbf{sounddevice 0.5.3}: Real-time audio I/O
    \item \textbf{Matplotlib 3.10.0}: Visualization
\end{itemize}

\subsection{Audio Processing Pipeline}

\begin{figure}[H]
    \centering
    % You can create a flowchart diagram here later
    \fbox{Processing Pipeline Diagram - To be added}
    \caption{Audio processing pipeline showing data flow from microphone through filtering stages to output.}
    \label{fig:pipeline}
\end{figure}

% ============================================================================
\section{Experimental Methodology}

\subsection{Test Environment}
\begin{itemize}
    \item \textbf{Hardware}: Laptop built-in microphone
    \item \textbf{Sample Rate}: 44.1 kHz
    \item \textbf{Bit Depth}: 16-bit PCM
    \item \textbf{Frame Size}: 2048 samples ($\approx$ 46 ms)
    \item \textbf{Recording Duration}: 5--10 seconds per test
\end{itemize}

\subsection{Performance Metrics}

\subsubsection{Signal-to-Noise Ratio (SNR)}
\begin{equation}
    \text{SNR} = 10 \log_{10} \left( \frac{P_{signal}}{P_{noise}} \right) \text{ dB}
\end{equation}

SNR improvement ($\Delta$SNR) measures filter effectiveness:
\begin{equation}
    \Delta\text{SNR} = \text{SNR}_{filtered} - \text{SNR}_{original}
\end{equation}

\subsection{Test Procedures}

\subsubsection{Phase A0: Baseline Verification}
\begin{enumerate}
    \item Record 5 seconds of speech
    \item Verify real-time waveform visualization
    \item Confirm FFT spectrum accuracy
    \item Validate WAV file export
\end{enumerate}

\subsubsection{Phase A1: Filter Testing}
\begin{enumerate}
    \item \textbf{Band-pass test}: Record speech in quiet environment
    \item \textbf{Notch test}: Record near 60 Hz noise source (charger)
    \item \textbf{Combined test}: Apply both filters sequentially
    \item Compare before/after spectrum and spectrogram
    \item Calculate SNR improvement
\end{enumerate}

% ============================================================================
\section{Results}

\subsection{Phase A0: Baseline System}

\subsubsection{Real-Time Visualization}
The system successfully captures and visualizes audio in real-time:

\begin{figure}[H]
    \centering
    \includegraphics[width=0.9\textwidth]{results/plots/phase_a0_realtime.png}
    \caption{Real-time waveform (top) and FFT spectrum (bottom) during recording.}
    \label{fig:a0_realtime}
\end{figure}

% ADD YOUR SCREENSHOT FROM THE TEST HERE
% Replace the filename above with the actual screenshot

\subsubsection{Audio Analysis}
\begin{figure}[H]
    \centering
    \includegraphics[width=0.9\textwidth]{results/plots/test_recording_a0_analysis.png}
    \caption{Spectrum (top) and spectrogram (bottom) of recorded speech sample showing frequency content and time-varying characteristics.}
    \label{fig:a0_analysis}
\end{figure}

\textbf{Observations:}
\begin{itemize}
    \item Strong low-frequency energy (0--1000 Hz) indicating room noise and voice fundamentals
    \item Speech energy concentrated primarily in 200--8000 Hz range with visible harmonics
    \item Spectrogram clearly shows 4--5 distinct speech events at approximately 1, 2, 3, and 4 seconds
    \item Bright green/yellow bands at low frequencies (0--500 Hz) indicate constant background noise floor
    \item Cyan/blue bursts extending up to $\sim$8 kHz represent speech content with good spectral resolution
    \item Natural roll-off above 8000 Hz as expected for speech signals
    \item Visible formant structure and harmonic content in the spectrogram
    \item System successfully captures full audio bandwidth up to Nyquist frequency (22.05 kHz)
\end{itemize}

\subsection{Phase A1: Baseline Filters}

\subsubsection{Filter Frequency Response}

\begin{figure}[H]
    \centering
    \begin{subfigure}[b]{0.48\textwidth}
        \includegraphics[width=\textwidth]{results/plots/a1_bandpass_response.png}
        \caption{Band-pass filter (80--8000 Hz)}
    \end{subfigure}
    \hfill
    \begin{subfigure}[b]{0.48\textwidth}
        \includegraphics[width=\textwidth]{results/plots/a1_notch_response.png}
        \caption{Notch filter (60 Hz)}
    \end{subfigure}
    \caption{Theoretical frequency responses of designed filters.}
    \label{fig:filter_responses}
\end{figure}

\subsubsection{Band-Pass Filter Results}

\begin{figure}[H]
    \centering
    \includegraphics[width=0.95\textwidth]{results/plots/a1_bandpass_comparison.png}
    \caption{Band-pass filter comparison showing spectrum (top row) and spectrogram (bottom row) for original (left) vs. filtered (right) audio. The filter effectively removes low-frequency rumble below 80 Hz and high-frequency noise above 8000 Hz while preserving speech content.}
    \label{fig:bandpass_comparison}
\end{figure}

\textbf{Qualitative Analysis:}
\begin{itemize}
    \item Spectrum shows sharp attenuation below 80 Hz and above 8000 Hz
    \item Spectrogram displays cleaner speech signal with reduced background noise
    \item Speech intelligibility and formant structure preserved in passband
    \item Visual confirmation of 5th-order Butterworth filter characteristics
\end{itemize}

\begin{table}[H]
\centering
\begin{tabular}{lcc}
\toprule
Metric & Original & Band-Pass Filtered \\
\midrule
Filter Type & --- & 5th-order Butterworth \\
Passband & Full spectrum & 80--8000 Hz \\
Attenuation & --- & $>$40 dB (stopband) \\
\bottomrule
\end{tabular}
\caption{Band-pass filter configuration and characteristics.}
\label{tab:bandpass_metrics}
\end{table}

\subsubsection{Notch Filter Results}

\begin{figure}[H]
    \centering
    \includegraphics[width=0.95\textwidth]{results/plots/a1_notch_comparison.png}
    \caption{Notch filter comparison showing 60 Hz hum removal.}
    \label{fig:notch_comparison}
\end{figure}

\textbf{Observations:}
\begin{itemize}
    \item Notch filter successfully targets 60 Hz and 120 Hz frequencies
    \item Very narrow bandwidth (Q = 30) ensures surgical precision
    \item Speech content remains completely unaffected due to high selectivity
    \item Filter response shows $>$20 dB attenuation at target frequencies
    \item Effectiveness depends on presence of 60 Hz hum in original recording
    \item Complementary to band-pass filtering for comprehensive noise reduction
\end{itemize}

\subsubsection{Combined Filtering}

The combined filtering approach applies both band-pass and notch filters sequentially to maximize noise reduction while preserving speech quality.

\begin{figure}[H]
    \centering
    \includegraphics[width=0.95\textwidth]{results/plots/a1_combined_comparison.png}
    \caption{Combined band-pass and notch filtering results showing sequential application of both filter types. Left column shows original audio; right column shows filtered output with both techniques applied.}
    \label{fig:combined_comparison}
\end{figure}

\textbf{Processing Pipeline:}
\begin{enumerate}
    \item Apply 5th-order Butterworth band-pass filter (80--8000 Hz)
    \item Apply cascaded IIR notch filters at 60 Hz and 120 Hz
    \item Result: Broadband noise reduction + tonal interference removal
\end{enumerate}

\textbf{Observations:}
\begin{itemize}
    \item Combined approach successfully removes both broadband and tonal noise components
    \item No observable filter interaction or instability in sequential application
    \item Spectrum shows clean passband with both edge attenuation and notch characteristics
    \item Speech quality maintained while achieving maximum noise reduction
    \item Spectrogram demonstrates effective removal of constant low-frequency noise floor
    \item Suitable for real-world applications where multiple noise types coexist
\end{itemize}

\textbf{Test Methodology Note:}
All Phase A1 tests were conducted with different audio samples (singing and vocal exclamations) to verify filter performance across varying input characteristics and dynamic ranges.

% ============================================================================
\subsection{Phase A2: Advanced Noise Reduction}

Phase A2 implements adaptive noise reduction algorithms that estimate and remove noise from audio signals using frequency-domain processing.

\subsubsection{Spectral Subtraction}

Spectral subtraction estimates noise during low-energy (silent) frames and subtracts it from the signal spectrum:

\begin{equation}
    |\hat{X}[k]| = \max(|Y[k]| - \alpha|N[k]|, \beta|Y[k]|)
\end{equation}

where $\alpha$ is the over-subtraction factor and $\beta$ is the spectral floor.

\begin{figure}[H]
    \centering
    \includegraphics[width=0.95\textwidth]{results/plots/a2_spectral_subtraction_comparison.png}
    \caption{Spectral subtraction results ($\alpha=2.0$, $\beta=0.02$) showing before (left) and after (right) noise reduction. Top row: magnitude spectra; bottom row: spectrograms.}
    \label{fig:spectral_subtraction}
\end{figure}

\textbf{Key Observations:}
\begin{itemize}
    \item Algorithm successfully identifies and removes noise during silent periods
    \item Over-subtraction factor $\alpha=2.0$ provides aggressive noise reduction
    \item Spectral floor $\beta=0.02$ prevents over-suppression artifacts (musical noise)
    \item Overlap-add reconstruction with Hann windowing ensures smooth transitions
    \item Effective for stationary noise (constant background hum, HVAC noise)
    \item Some residual ``musical noise'' artifacts may occur with aggressive parameters
\end{itemize}

\subsubsection{Adaptive Spectral Subtraction}

The adaptive variant adjusts the over-subtraction factor based on local SNR:

\begin{equation}
    \alpha(t) = \alpha_{max} - (\alpha_{max} - \alpha_{min}) \cdot \min\left(\frac{\text{SNR}(t)}{10}, 1\right)
\end{equation}

\begin{figure}[H]
    \centering
    \includegraphics[width=0.95\textwidth]{results/plots/a2_adaptive_spectral_comparison.png}
    \caption{Adaptive spectral subtraction with SNR-based parameter control ($\alpha$ range: 1.0--4.0).}
    \label{fig:adaptive_spectral}
\end{figure}

\textbf{Advantages over Basic Spectral Subtraction:}
\begin{itemize}
    \item Adapts to varying noise levels across different sections of audio
    \item More aggressive ($\alpha \approx 4.0$) during high-noise regions
    \item Gentler ($\alpha \approx 1.0$) during clean speech sections
    \item Reduces over-processing of already-clean signal
    \item Better preservation of speech quality compared to fixed parameters
\end{itemize}

\subsubsection{Wiener Filter}

The Wiener filter computes optimal gain in the MMSE (Minimum Mean-Square Error) sense:

\begin{equation}
    H[k] = \frac{S_{xx}[k]}{S_{xx}[k] + S_{nn}[k]}
\end{equation}

where $S_{xx}$ is the estimated signal PSD and $S_{nn}$ is the noise PSD.

\begin{figure}[H]
    \centering
    \includegraphics[width=0.95\textwidth]{results/plots/a2_wiener_comparison.png}
    \caption{Wiener filter results showing optimal MMSE-based noise reduction.}
    \label{fig:wiener}
\end{figure}

\textbf{Performance Characteristics:}
\begin{itemize}
    \item Theoretically optimal for Gaussian noise under MMSE criterion
    \item Smoother gain function reduces musical noise compared to spectral subtraction
    \item Gain smoothing (exponential averaging) further reduces artifacts
    \item Excellent preservation of speech naturalness
    \item Computationally efficient (single-pass frequency-domain processing)
    \item Works well for both stationary and slowly-varying noise
\end{itemize}

\subsubsection{Adaptive Wiener Filter}

Adaptive variant includes real-time noise PSD tracking:

\begin{equation}
    S_{nn}[k,t] = \alpha_n \cdot S_{nn}[k,t-1] + (1-\alpha_n) \cdot \min(S_{yy}[k,t], S_{nn}[k,t-1])
\end{equation}

\begin{figure}[H]
    \centering
    \includegraphics[width=0.95\textwidth]{results/plots/a2_adaptive_wiener_comparison.png}
    \caption{Adaptive Wiener filter with continuous noise tracking and minimum gain floor.}
    \label{fig:adaptive_wiener}
\end{figure}

\textbf{Enhancements:}
\begin{itemize}
    \item Tracks time-varying noise characteristics
    \item Adapts to changing noise environments during recording
    \item Minimum gain floor ($-60$ dB) prevents complete signal suppression
    \item Better performance in non-stationary noise scenarios
    \item Particularly effective for real-world applications with varying noise
\end{itemize}

\subsubsection{Method Comparison}

All four methods were applied to identical audio samples for direct comparison:

\begin{figure}[H]
    \centering
    \includegraphics[width=0.95\textwidth]{results/plots/a2_method_comparison.png}
    \caption{Side-by-side comparison of all four Phase A2 noise reduction methods showing original (left) vs. filtered (right) spectra for each technique.}
    \label{fig:method_comparison}
\end{figure}

\textbf{Comparative Analysis:}
\begin{itemize}
    \item \textbf{Spectral Subtraction}: Fastest processing, most aggressive, some artifacts
    \item \textbf{Adaptive Spectral}: Better quality than basic, adapts to signal characteristics
    \item \textbf{Wiener Filter}: Best overall quality, minimal artifacts, natural sound
    \item \textbf{Adaptive Wiener}: Most robust to varying noise, best for real-world use
    \item All methods achieve significant SNR improvement over baseline
    \item Wiener-based methods generally outperform spectral subtraction in perceptual quality
    \item Trade-off between processing complexity and output quality
\end{itemize}

\textbf{Recommended Applications:}
\begin{itemize}
    \item \textbf{Real-time communication}: Adaptive Wiener (best quality + adaptivity)
    \item \textbf{Post-processing/archival}: Wiener filter (optimal quality)
    \item \textbf{Low-latency requirements}: Spectral subtraction (fastest)
    \item \textbf{Varying noise environments}: Adaptive methods (both spectral and Wiener)
\end{itemize}

% ============================================================================
\subsection{Phase B: Simulated ANC (FxLMS)}
% To be filled in after implementation

% ============================================================================
\section{Discussion}

\subsection{Analysis of Results}

\subsubsection{Filter Performance}
The implemented filtering system demonstrates practical noise reduction capabilities:

\textbf{Phase A0 Achievements:}
\begin{itemize}
    \item Successfully established real-time audio I/O pipeline at 44.1 kHz
    \item FFT implementation provides accurate frequency-domain representation
    \item Spectrogram visualization clearly reveals time-varying speech characteristics
    \item System captures full audio bandwidth with no aliasing artifacts
    \item Background noise floor clearly distinguishable from speech events
\end{itemize}

\textbf{Phase A1 Filter Analysis:}
\begin{itemize}
    \item \textbf{Band-pass filter:} Effectively removes out-of-band noise while preserving speech intelligibility
        \begin{itemize}
            \item 5th-order design provides $>$40 dB stopband attenuation
            \item Flat passband response maintains speech quality
            \item Zero-phase filtering avoids temporal distortion
            \item Particularly effective for removing low-frequency rumble (HVAC, handling noise)
        \end{itemize}
    \item \textbf{Notch filters:} Provide surgical removal of tonal interference
        \begin{itemize}
            \item High Q-factor (30) ensures narrow bandwidth
            \item Minimal impact on surrounding frequencies
            \item Effectiveness verified through visual spectrum analysis
            \item Cascaded approach successfully targets multiple harmonics
        \end{itemize}
    \item \textbf{Combined approach:} Sequential application demonstrates complementary benefits
        \begin{itemize}
            \item Band-pass first removes broadband noise
            \item Notch filters then target residual tonal components
            \item No observable interaction between filter stages
        \end{itemize}
\end{itemize}

\subsubsection{Limitations}
Despite successful implementation, several limitations were identified:

\textbf{Fixed Filter Parameters:}
\begin{itemize}
    \item Filter cutoff frequencies are preset and do not adapt to varying noise conditions
    \item Band-pass range (80--8000 Hz) may be overly aggressive for some applications
        \begin{itemize}
            \item Low-frequency speech information below 80 Hz is discarded
            \item Some high-frequency consonants above 8000 Hz lost
        \end{itemize}
    \item Notch filter assumes 60 Hz interference (not universal in all regions)
\end{itemize}

\textbf{Processing Constraints:}
\begin{itemize}
    \item Current implementation uses record-process-play workflow (not truly real-time)
    \item Processing latency would be problematic for interactive applications
    \item No voice activity detection (VAD) - filters applied uniformly
    \item Single-channel (mono) processing only
\end{itemize}

\textbf{Noise Reduction Scope:}
\begin{itemize}
    \item Filters cannot distinguish between desired and undesired signals in same frequency band
    \item Non-stationary noise (e.g., sudden impacts) not effectively addressed
    \item Performance degrades when noise and speech spectra overlap significantly
    \item No adaptation to changing noise environments during recording
\end{itemize}

\subsection{Future Improvements}

\textbf{Immediate Enhancements (Phase A2):}
\begin{itemize}
    \item \textbf{Spectral Subtraction:} Adaptive noise spectrum estimation and subtraction
    \item \textbf{Wiener Filtering:} Optimal MMSE-based noise reduction
    \item \textbf{Adaptive thresholding:} Automatic parameter selection based on noise profile
\end{itemize}

\textbf{Advanced Features (Phase B):}
\begin{itemize}
    \item \textbf{FxLMS Algorithm:} Adaptive active noise cancellation simulation
    \item \textbf{Multi-channel processing:} Spatial filtering techniques
    \item \textbf{Machine learning:} Neural network-based denoising (e.g., RNNoise)
\end{itemize}

\textbf{System Improvements:}
\begin{itemize}
    \item True real-time processing with minimal latency ($<$30 ms)
    \item Voice activity detection (VAD) for selective processing
    \item Adaptive filter parameter tuning based on input characteristics
    \item Hardware implementation on embedded systems (Raspberry Pi, DSP chips)
    \item Integration with communication systems (VoIP, conferencing)
\end{itemize}

% ============================================================================
\section{Conclusion}

This project successfully implements a comprehensive audio noise reduction system demonstrating both classical and modern DSP techniques. The system achieves measurable SNR improvements through multiple filtering approaches:

\begin{itemize}
    \item Phase A0 establishes real-time audio processing pipeline
    \item Phase A1 baseline filters provide [X] dB average improvement
    \item System demonstrates practical application of signal processing theory
    \item Implementation serves as foundation for advanced techniques
\end{itemize}

The modular architecture enables easy extension and experimentation with different algorithms. All source code and documentation are available on GitHub: \url{https://github.com/jahuytee/Noise-cancellation-Signal-System}

% ============================================================================
\section{Appendix}

\subsection{Source Code Repository}
Complete source code: \url{https://github.com/jahuytee/Noise-cancellation-Signal-System}

\subsection{Key Functions}

\subsubsection{Band-Pass Filter Implementation}
\begin{lstlisting}
def apply_bandpass_filter(audio_data, sample_rate, 
                         lowcut=80, highcut=8000, order=5):
    nyquist = 0.5 * sample_rate
    low = lowcut / nyquist
    high = highcut / nyquist
    b, a = signal.butter(order, [low, high], btype='band')
    filtered_audio = signal.filtfilt(b, a, audio_data)
    return filtered_audio
\end{lstlisting}

\subsection{Hardware and Software Specifications}
\begin{itemize}
    \item \textbf{Operating System}: Windows 11
    \item \textbf{Python Version}: 3.10
    \item \textbf{Processor}: [Your CPU]
    \item \textbf{RAM}: [Your RAM]
    \item \textbf{Microphone}: Laptop built-in
\end{itemize}

\end{document}
